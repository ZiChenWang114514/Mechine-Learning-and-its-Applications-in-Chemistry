\documentclass{article}
\usepackage{graphicx} % Required for inserting images
\usepackage[utf8]{inputenc}

\usepackage{ctex}
\usepackage{amsfonts, amsmath, amsthm, amssymb, bm, graphicx, hyperref, mathrsfs, indentfirst}
\usepackage{hyperref}
\hypersetup{
    colorlinks=true, % 将colorlinks设置为true以启用彩色链接
    linkcolor=black, % 链接颜色为黑色
    citecolor=black, % 引用颜色为黑色
    urlcolor=black % URL颜色为黑色
}
\usepackage{geometry}
\geometry{left=2.54cm,right=2.54cm,top=3.18cm,bottom=3.18cm}
\usepackage{tikz}
\usepackage{bussproofs}

\usepackage{xcolor}
\usepackage[most]{tcolorbox}


\title{\huge{\textbf{MLChem\,作业二}}}
\author{\kaishu 2100011873 王子宸\\
        \kaishu 化学与分子工程学院}
\date{\kaishu \today}
\begin{document}

\maketitle

\section{题目1}

Sigmoid函数:
\begin{equation}\label{eq1}
    \sigma(a) = \dfrac{1}{1+e^{-a}}
\end{equation}

判断其对称性质,根据 \eqref{eq1},有
\begin{align*}
    \sigma(-a) = \dfrac{1}{1+e^{a}} = \dfrac{e^{-a}}{e^{-a}+1} = 1 - \dfrac{1}{1+e^{-a}} = 1 - \sigma(a)
\end{align*}

考虑其逆函数,令:
\begin{equation*}
    y = \sigma(x) = \dfrac{1}{1+e^{-x}}
\end{equation*}
有:
\begin{align*}
    e^{-x} &= \dfrac{1}{y} - 1\\
    -x &= \ln\dfrac{1-y}{y}\\
    \therefore x = \sigma^{-1}(x) &= \ln\dfrac{y}{1-y}
\end{align*}

\section{题目2}

在逻辑回归中,预测的线性函数为:
\begin{equation}
z = w_0 + w_1x_1 + w_2x_2
\end{equation}

sigmoid函数定义为:
\begin{equation}
\sigma(z) = \dfrac{1}{1 + e^{-z}}
\end{equation}

给定学习到的参数 $w_0=2, w_1=-2, w_2=1$,我们可以先计算 $z$ 的值。

\begin{enumerate}
    \item 当 $\mathbf{x} =(0.5,1)$ 时:
    \begin{align*}
    z &= 2 - 2\times 0.5 + 1\times 1 \\
    &= 2 - 1 + 1 \\
    &= 2
    \end{align*}
    然后,计算 $\sigma(z)$ 的值:
    \begin{equation}
    \sigma(2) = \dfrac{1}{1 + e^{-2}} \approx 0.8808
    \end{equation}

    \item 当 $\mathbf{x} =(1,-1)$ 时:
    \begin{align*}
    z &= 2 - 2(1) + 1(-1) \\
    &= 2 - 2 - 1 \\
    &= -1
    \end{align*}
    然后,计算 $\sigma(z)$ 的值:
    \begin{equation}
    \sigma(-1) = \dfrac{1}{1 + e^{1}} \approx 0.2689
    \end{equation}
\end{enumerate}

所以,在 $\mathbf{x} =(0.5,1)$ 时逻辑函数的值为 0.8808,而在 $\mathbf{x} =(1,-1)$ 时的值为 0.2689。

接下来,关于决策面:
在二维平面上,我们可以考虑 $z = 0$ 的情况,因为这是两个分类之间的边界,即 $\sigma(z) = 0.5$。给定 $z = w_0 + w_1x_1 + w_2x_2 = 0$,我们得到的决策面方程为:
\begin{equation}
x_2 = 2x_1 - 2
\end{equation}
这就是在 $(x_1,x_2)$ 平面上的决策面。


\section{题目3}

对于二值的归类问题,逻辑回归表示的概率:
\begin{equation}\label{eq2}
    \begin{aligned}
    & p(y=1 \mid \boldsymbol{x})=\dfrac{e^{\boldsymbol{w}^{\mathrm{T}} \boldsymbol{x}+b}}{1+e^{\boldsymbol{w}^{\mathrm{T}} \boldsymbol{x}+b}} \\
    & p(y=0 \mid \boldsymbol{x})=\dfrac{1}{1+e^{\boldsymbol{w}^{\mathrm{T}} \boldsymbol{x}+b}}
    \end{aligned}
\end{equation}


根据最大似然估计,逻辑回归的对数似然函数为:
\begin{equation}
    \ell(\boldsymbol{w}, b)=\sum_{i=1}^m \ln p\left(y_i \mid \boldsymbol{x}_i ; \boldsymbol{w}, b\right)
\end{equation}
其中,将$\boldsymbol{w}^{\mathrm{T}} \boldsymbol{x}+b$记为$\mathbf{w}^{\mathrm{T}}x$,损失函数取为对数似然函数的负值,得到损失函数:
\begin{equation*}
    \begin{aligned}
E(\mathbf{w}) & =-\sum_{i=1}^m \ln \left(\left[p_1\left(\hat{\boldsymbol{x}}_i ; \mathbf{w}\right)\right]^{y_i}\left[p_0\left(\hat{\boldsymbol{x}}_i ; \mathbf{w}\right)\right]^{1-y_i}\right) \\
& =-\sum_{i=1}^m\left[y_i \ln \left(p_1\left(\hat{\boldsymbol{x}}_i ; \mathbf{w}\right)\right)+\left(1-y_i\right) \ln \left(p_0\left(\hat{\boldsymbol{x}}_i ; \mathbf{w}\right)\right)\right] \\
& =-\sum_{i=1}^m\left\{y_i\left[\ln \left(p_1\left(\hat{\boldsymbol{x}}_i ; \mathbf{w}\right)\right)-\ln \left(p_0\left(\hat{\boldsymbol{x}}_i ; \mathbf{w}\right)\right)\right]+\ln \left(p_0\left(\hat{\boldsymbol{x}}_i ; \mathbf{w}\right)\right)\right\} \\
& =-\sum_{i=1}^m\left[y_i \ln \left(\dfrac{p_1\left(\hat{\boldsymbol{x}}_i ; \mathbf{w}\right)}{p_0\left(\hat{\boldsymbol{x}}_i ; \mathbf{w}\right)}\right)+\ln \left(p_0\left(\hat{\boldsymbol{x}}_i ; \mathbf{w}\right)\right)\right] \\
& =-\sum_{i=1}^m\left[y_i \ln \left(e^{\mathbf{w}^{\mathrm{T}} \hat{\boldsymbol{x}}_i}\right)+\ln \left(\dfrac{1}{1+e^{\mathbf{w}^{\mathrm{T}} \hat{\boldsymbol{x}}_i}}\right)\right] \\
& =-\sum_{i=1}^m\left(y_i \mathbf{w}^{\mathrm{T}} \hat{\boldsymbol{x}}_i-\ln \left(1+e^{\mathbf{w}^{\mathrm{T}} \hat{\boldsymbol{x}}_i}\right)\right)
\end{aligned}
\end{equation*}

对损失函数求一阶导数,得到损失函数的梯度:

\begin{equation*}
    \begin{aligned}
        \nabla E(\mathrm{w})=\dfrac{\partial E(\mathbf{w})}{\partial \mathbf{w}} & =\dfrac{\partial \sum_{i=1}^m\left(\ln \left(1+e^{\mathbf{w}^{\mathrm{T}} \hat{\boldsymbol{x}}_i}\right) - y_i \mathbf{w}^{\mathrm{T}} \hat{\boldsymbol{x}}_i\right)}{\partial \mathbf{w}} \\
        & =\sum_{i=1}^m\left(\dfrac{\partial \ln \left(1+e^{\mathbf{w}^{\mathrm{T}} \hat{\boldsymbol{x}}_i}\right)}{\partial \mathbf{w}}-\dfrac{\partial\left(y_i \mathbf{w}^{\mathrm{T}} \hat{\boldsymbol{x}}_i\right)}{\partial \mathbf{w}}\right) \\
        & =\sum_{i=1}^m\left(\dfrac{1}{1+e^{\mathbf{w}^{\mathrm{T}} \hat{\boldsymbol{x}}_i}} \cdot \hat{\boldsymbol{x}}_i e^{\mathbf{w}^{\mathrm{T}} \hat{\boldsymbol{x}}_i}-y_i \hat{\boldsymbol{x}}_i\right) \\
        & =\sum_{i=1}^m \hat{\boldsymbol{x}}_i\left(\dfrac{e^{\mathbf{w}^{\mathrm{T}} \hat{\boldsymbol{x}}_i}}{1+e^{\mathbf{w}^{\mathrm{T}} \hat{\boldsymbol{x}}_i}}-y_i\right) \\
        & =\sum_{i=1}^m \hat{\boldsymbol{x}}_i\left(p_1\left(\hat{\boldsymbol{x}}_i ; \mathbf{w}\right)-y_i\right)
    \end{aligned}
\end{equation*}

对损失函数求二阶导数:
\begin{equation*}
    \begin{aligned}
        \dfrac{\partial^2 E(\mathrm{w})}{\partial \mathrm{w} \partial \mathrm{w}^{\mathrm{T}}} & =\dfrac{\partial \sum_{i=1}^m \hat{\boldsymbol{x}}_i\left(\dfrac{e^{\mathrm{w}^{\mathrm{T}} \hat{x}_i}}{1+e^{\mathrm{T}} \hat{\boldsymbol{x}}_i}-y_i\right)}{\partial \mathrm{w}^{\mathrm{T}}} \\
        & =\sum_{i=1}^m \hat{\boldsymbol{x}}_i \dfrac{\partial\left(\dfrac{e^{\mathrm{w}^{\mathrm{T}} \hat{\hat{x}}_i}}{1+e^{\mathrm{w}^{\mathrm{T}} \hat{\boldsymbol{x}}_i}}-y_i\right)}{\partial \mathrm{w}^{\mathrm{T}}} \\
        & =\sum_{i=1}^m \hat{\boldsymbol{x}}_i\left(\dfrac{\partial\left(\dfrac{e^{\mathrm{w}^{\mathrm{T}} \hat{\boldsymbol{x}}_i}}{1+e^{\mathrm{w}^{\mathrm{T}} \hat{\boldsymbol{x}}_i}}\right)}{\partial \mathrm{w}^{\mathrm{T}}}-\dfrac{\partial y_i}{\partial \mathrm{w}^{\mathrm{T}}}\right) \\
        & =\sum_{i=1}^m \hat{\boldsymbol{x}}_i \cdot \dfrac{\partial\left(\dfrac{e^{\mathrm{w}^{\mathrm{T}} \hat{\boldsymbol{x}}_i}}{1+e^{\mathrm{w}^{\mathrm{T}} \hat{\boldsymbol{x}}_i}}\right)}{\partial \mathrm{w}^{\mathrm{T}}}\\
        & =\sum_{i=1}^m \hat{\boldsymbol{x}}_i\cdot\left(\dfrac{\hat{\boldsymbol{x}}_i^{\mathrm{T}}e^{\mathrm{w}^{\mathrm{T}} \hat{\boldsymbol{x}}_i}}{1+e^{\mathrm{w}^{\mathrm{T}} \hat{\boldsymbol{x}}_i}} - \dfrac{e^{\mathrm{w}^{\mathrm{T}} \hat{\boldsymbol{x}}_i} \cdot \hat{\boldsymbol{x}}_i^{\mathrm{T}}e^{\mathrm{w}^{\mathrm{T}} \hat{\boldsymbol{x}}_i}}{\left(1+e^{\mathrm{w}^{\mathrm{T}} \hat{\boldsymbol{x}}_i}\right)^2} \right)\\
        & =\sum_{i=1}^m \hat{\boldsymbol{x}}_i\cdot \dfrac{\hat{\boldsymbol{x}}_i^{\mathrm{T}}e^{\mathrm{w}^{\mathrm{T}} \hat{\boldsymbol{x}}_i}\left(1+e^{\mathrm{w}^{\mathrm{T}} \hat{\boldsymbol{x}}_i}\right) - e^{\mathrm{w}^{\mathrm{T}} \hat{\boldsymbol{x}}_i} \cdot \hat{\boldsymbol{x}}_i^{\mathrm{T}}e^{\mathrm{w}^{\mathrm{T}} \hat{\boldsymbol{x}}_i}}{\left(1+e^{\mathrm{w}^{\mathrm{T}} \hat{\boldsymbol{x}}_i}\right)^2} \\
        & =\sum_{i=1}^m \hat{\boldsymbol{x}}_i\cdot \dfrac{\hat{\boldsymbol{x}}_i^{\mathrm{T}}e^{\mathrm{w}^{\mathrm{T}} \hat{\boldsymbol{x}}_i}}{\left(1+e^{\mathrm{w}^{\mathrm{T}} \hat{\boldsymbol{x}}_i}\right)^2} \\
        & = \sum_{i=1}^m \hat{\boldsymbol{x}}_i \hat{\boldsymbol{x}}_i^{\mathrm{T}}p_1\left(\hat{\boldsymbol{x}}_i ; \mathbf{w}\right)\left(1-p_1\left(\hat{\boldsymbol{x}}_i ; \mathbf{w}\right)\right)
    \end{aligned}
\end{equation*}
\end{document}
