\documentclass{article}
\usepackage{graphicx} % Required for inserting images
\usepackage[utf8]{inputenc}

\usepackage{ctex}
\usepackage{amsfonts, amsmath, amsthm, amssymb, bm, graphicx, hyperref, mathrsfs, indentfirst}
\usepackage{geometry}
\geometry{left=2.54cm,right=2.54cm,top=3.18cm,bottom=3.18cm}
\usepackage{tikz}
\usepackage{bussproofs}

\usepackage{xcolor}
\usepackage[most]{tcolorbox}


\title{\huge{\textbf{MLChem\,作业一}}}
\author{\kaishu 2100011873 王子宸\\
        \kaishu 化学与分子工程学院}
\date{\kaishu \today}
\begin{document}

\maketitle

\section{题目1}

\subsection{}
已阅
\subsection{}

已知:$s\sim \mathcal{N}(\mu,\sigma^2), t\sim \mathcal{N}(\mu,\sigma^2), s = f(t) = at + b$,令:
\begin{align*}
    y &= s - \mu,\quad x\sim \mathcal{N}(0,\sigma^2)\\
    x &= t - \mu,\quad y\sim \mathcal{N}(0,\sigma^2)
\end{align*}
\begin{align*}
    \therefore  y& + \mu = f(t) = f(x) = a(x + \mu) + b\\
    \therefore  y& = ax + [\mu(a-1)+b]
\end{align*}
有线性回归,在$y = ax + [\mu(a-1)+b]$中,
\begin{align*}
    &a = \frac{\langle xy \rangle-\langle x\rangle\langle y\rangle}{\langle x^2\rangle-\langle x\rangle^2} = \frac{r\sigma^2}{\sigma^2} = r\\
    &[\mu(a-1)+b] = \frac{\langle x^2\rangle\langle y\rangle-\langle x\rangle\langle xy\rangle}{\langle x^2\rangle-\langle x\rangle^2} = 0\\
    \therefore \,&b = \mu(1-a)\\
\end{align*}
最终,得到:
\begin{align*}
    y &= rx\\
    s &= rt + \mu(1-a)
\end{align*}
结论:
\begin{itemize}
    \item 如果$x>0$(即$t>\mu$),平均而言,父亲的身高高于儿子的身高;
    \item 如果$x=0$(即$t=\mu$),平均而言,父亲的身高与儿子的身高相同;
    \item 如果$x<0$(即$t<\mu$),平均而言,父亲的身高低于儿子的身高;
\end{itemize}

\subsection{}
已知:$s\sim \mathcal{N}(\mu,\sigma^2), t\sim \mathcal{N}(\mu,\sigma^2), t = f(s) = as + b$,同理,令:
\begin{align*}
    x &= s - \mu,\quad x\sim \mathcal{N}(0,\sigma^2)\\
    y &= t - \mu,\quad y\sim \mathcal{N}(0,\sigma^2)
\end{align*}
\begin{align*}
    \therefore  y& + \mu = f(s) = f(x) = a(x + \mu) + b\\
    \therefore  y& = ax + [\mu(a-1)+b]
\end{align*}
有线性回归,在$y = ax + [\mu(a-1)+b]$中,
\begin{align*}
    &a = \frac{\langle xy \rangle-\langle x\rangle\langle y\rangle}{\langle x^2\rangle-\langle x\rangle^2} = \frac{r\sigma^2}{\sigma^2} = r\\
    &[\mu(a-1)+b] = \frac{\langle x^2\rangle\langle y\rangle-\langle x\rangle\langle xy\rangle}{\langle x^2\rangle-\langle x\rangle^2} = 0\\
    \therefore \,&b = \mu(1-a)\\
\end{align*}
最终,得到:
\begin{align*}
    y &= rx\\
    t &= rs + \mu(1-a)
\end{align*}
结论:
\begin{itemize}
    \item 如果$x>0$(即$s>\mu$),平均而言,儿子的身高高于父亲的身高;
    \item 如果$x=0$(即$s=\mu$),平均而言,儿子的身高与父亲的身高相同;
    \item 如果$x<0$(即$s<\mu$),平均而言,儿子的身高低于父亲的身高;
\end{itemize}

\subsection{}
“父亲平均矮于儿子”与“儿子平均矮于父亲”能够同时成立:因为这两句话中,“父亲”与“儿子”是完全不同的样本集。“父亲平均矮于儿子”中,“儿子”是平均身高较高的样品集,“父亲”考虑的是这个样品集中儿子的父亲;而“儿子平均矮于父亲”中,“父亲”是平均身高较高的样品集,“儿子”考虑的是这个样品集中父亲的儿子。

\section{题目2}
\begin{itemize}
    \item 根据逆概公式(Bayes公式):
    \begin{align*}
        P(\text{狄青钱}|\text{四次正面}) &= \frac{P(\text{三次正面}|\text{狄青钱})P(\text{狄青钱})}{P(\text{三次正面})}\\
        &= \frac{P(\text{三次正面}|\text{狄青钱})P(\text{狄青钱})}{P(\text{三次正面}|\text{狄青钱})P(\text{狄青钱})+P(\text{三次正面}|\text{正常铜钱})P(\text{正常铜钱})}\\
        &= \frac{1\times \dfrac{50}{100}}{1\times \dfrac{50}{100}+\left(\dfrac{1}{2}\right)^3\times \dfrac{50}{100}}\\
        & = \frac89
    \end{align*}
    \item 如果第四次仍然为正面,同理,根据逆概公式:
    \begin{align*}
        P(\text{狄青钱}|\text{四次正面}) &= \frac{P(\text{四次正面}|\text{狄青钱})P(\text{狄青钱})}{P(\text{四次正面})}\\
        &= \frac{P(\text{四次正面}|\text{狄青钱})P(\text{狄青钱})}{P(\text{四次正面}|\text{狄青钱})P(\text{狄青钱})+P(\text{四次正面}|\text{正常铜钱})P(\text{正常铜钱})}\\
        &= \frac{1\times \dfrac{50}{100}}{1\times \dfrac{50}{100}+\left(\dfrac{1}{2}\right)^4\times \dfrac{50}{100}}\\
        & = \frac{15}{16}
    \end{align*}
    \item 如果第四次为反面,显然地:
    \begin{align*}
        P(\text{狄青钱}|\text{三次正面一次反面}) &= 0\\
        P(\text{正常铜钱}|\text{三次正面一次反面}) &= 1
    \end{align*}
\end{itemize}
\end{document}
